

\chapter{Produktdaten}
Ein Projekt fasst folgende Daten zusammen:
	\begin{itemize}
		\item Projektname und -beschreibung
		%  Wer hat eig. über das gesamte Pflichtenheft hinweg dieses höchst unpassende Wort eines
		% Testsignals eingeführt ?
		\item Dateipfad zu Referenzvideos als .yuv-Dateien (optional - es ist auch möglich nur mit generierten
		 Testsignalen zu arbeiten)
		 % Die Einstellungen eines filters gehören nicht zum Projekt
		 % Dieses Könnte Probleme nach sich ziehen.
		 % Z.b. exportierst du ein Projekt mit bestimmten Einstellungen für einen Filter
		 % auf eine andere Maschine. Wenn diese andere OQAT Installation aber nicht alle diese Filter
		 % hat kann es haarig werden.
		 % deshalb würde ich Vorschlagen Filtereinstellungen gehören nicht ins Projekt.
		\item Reihenfolge und Parameter der angewendeten Filter oder Testsignale
		% Testsignale vs. Videos ..... ich hab nicht mal mehr die Lust es zu ändern.
	        \item Testvideos als .yuv-Dateien, die durch Filter aus den Referenzvideos oder Testsignalen
	         generiert wurden
	         % Verflucht optional ;-)
	         % wird wahrscheinlich rausfliegen
	        \item Angaben (Pfad, Parameter) zur Ansteuerung des zu testenden Videobearbeitungswerkzeugs 
	        oder Encoders (optional)
		\item Ergebnisse der Analysedurchläufe, abhängig von der gewählten Metrik
		%was soll denn so ein Errorlog an Sinnvollen daten beinhalten ?
		% Bsp: ">Konnte Vorgang nicht ausführen"<.. Wir unser System ist im weitesten Sinne homogen,
		% Wir haben fast nur Umgang mit YUV Dateien, mir fällt keine sinnvolle Errormeldung, die der
		% Nutzer aufbewahren möchte, ein.
		\item Errorlogs
	\end{itemize}
	% c# hat zwei (native) Möglichkeiten um Anwendungsdaten abzulegen, XML und als Binary.
	% Warum hat sich der Autor dieser Zeilen für das eine und gegen das andere entschieden.
Die Einstellungen werden in einer XML-Datei gespeichert. Analyse-Ergebnisse werden in separaten Dateien
 gespeichert. Die generierten Testvideos werden vom Programm im Projektunterordner "Testvideos" abgelegt.


% brauchen wir auch eine Beschreibung des YUV-Format?
% Ja !
