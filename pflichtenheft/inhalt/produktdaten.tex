

\chapter{Produktdaten}
Ein Projekt fasst folgende Daten zusammen:
	\begin{enumerate}
	        \item Projektname und -beschreibung
		\item Dateipfad zu Referenzvideos als .yuv-Dateien (optional - es ist auch möglich nur mit generierten Testsignalen zu arbeiten)
		\item Reihenfolge und Parameter der angewendeten Filter oder Testsignale
	        \item Testvideos als .yuv-Dateien, die durch Filter aus den Referenzvideos oder Testsignalen generiert wurden
	        \item Angaben (Pfad, Parameter) zur Ansteuerung des zu testenden Videobearbeitungswerkzeugs oder Encoders (optional)
		\item Ergebnisse der Analysedurchläufe, abhängig von der gewählten Metrik
		\item Errorlogs
	\end{enumerate}
Die Einstellungen werden in einer XML-Datei gespeichert. Analyse-Ergebnisse werden in separaten Dateien gespeichert. Die generierten Testvideos werden vom Programm im Projektunterordner "Testvideos" abgelegt.


% brauchen wir auch eine Beschreibung des YUV-Format?
