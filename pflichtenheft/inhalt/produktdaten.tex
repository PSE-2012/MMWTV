

\chapter{Produktdaten}
\setcounter{counterKriterien}{0}
Anwendungsdaten: \\
\nItem{PD} Globale Einstellungen \\
\nItem{PD} Filterdateien \\
Enthalten alle zur Verfügung stehenden Filter und relevante Einstellungen zu denen \\
% Z. B. wenn wir den Weichzeichner auf 20% einstellen, die Anwendung schliessen
% und dann die Anwendung wieder öffnen und den Weichzeichner verwenden wollen, ist es als default
% auf 20% eingestellt, so habe ich Einstellungen gemeint.
\nItem{PD} Metridateien \\
Enthalten alle zur Verfügung stehenden Analysemetriken. \\
\nItem{PD} Lokalisierungsdateien \\
Die Lokalisierungsdateien der Anwendung werden extern gespeichert. Die deutsche Sprachdatei wird standardmäßig bereitgestellt. \\
\nItem{PD} Projektdateien \\
Ein Projekt fasst folgende Daten zusammen:
\begin{itemize}
\item Projektname und -beschreibung
\item Referenzen zu den dem Projekt zugewiesenen Videos
\item Ergebnisse absolvierter Analysevorgänge werden in separaten Dateien gespeichert
\item Die Videodateien des Benutzers \\
% Die Einstellungen eines filters gehören nicht zum Projekt
% Dieses Könnte Probleme nach sich ziehen.
% Z.b. exportierst du ein Projekt mit bestimmten Einstellungen für einen Filter
% auf eine andere Maschine. Wenn diese andere OQAT Installation aber nicht alle diese Filter
% hat kann es haarig werden.
% deshalb würde ich Vorschlagen Filtereinstellungen gehören nicht ins Projekt.
Die durch Filtervorgänge generierten Videodateien werden vom Programm im Projektunterordner \emph{Testvideos} abgelegt.
\item Logdatei für den Projektunterordner \emph{Testvideos} \\
Enthält Informationen darüber, welche Filter verwendet wurden, um die im Ordner liegenden Videodateien zu erzeugen.
\end{itemize}