

\chapter{Produktdaten}
\setcounter{counterKriterien}{0}
Anwendungsdaten: \\
\nItem{PD} Globale Einstellungen \\
\nItem{PD} Filterdateien \\
Enthalten zur Verfügung stehende Filter und relevante Einstellungen zu diesen. \\
% Z. B. wenn wir den Weichzeichner auf 20% einstellen, die Anwendung schliessen
% und dann die Anwendung wieder öffnen und den Weichzeichner verwenden wollen, ist es als default
% auf 20% eingestellt, so habe ich Einstellungen gemeint.
\nItem{PD} Metridateien \\
Enthalten zur Verfügung stehende Analysemetriken. \\
\nItem{PD} Sprachdateien \\
Die Lokalisierungsdateien der Anwendung werden extern gespeichert. Die deutsche Sprachdatei wird standardmäßig bereitgestellt. \\
\nItem{PD} Projektdateien \\
Ein Projekt fasst folgende Daten zusammen:
\begin{itemize}
\item Projektname und -beschreibung
\item Referenzen zu dem Projekt zugewiesenen Videos
\item Ergebnisse absolvierter Analysevorgänge
\item Videodateien des Benutzers \\
% Die Einstellungen eines filters gehören nicht zum Projekt
% Dieses Könnte Probleme nach sich ziehen.
% Z.b. exportierst du ein Projekt mit bestimmten Einstellungen für einen Filter
% auf eine andere Maschine. Wenn diese andere OQAT Installation aber nicht alle diese Filter
% hat kann es haarig werden.
% deshalb würde ich Vorschlagen Filtereinstellungen gehören nicht ins Projekt.
Die durch Filtervorgänge generierten Videodateien werden vom Programm generiert und gespeichert.
\item Logdateien\\
Enthalten Informationen darüber, welche Filter verwendet wurden, um die im Ordner liegenden Videodateien zu erzeugen.
\end{itemize}