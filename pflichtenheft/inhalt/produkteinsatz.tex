\chapter{Produkteinsatz}
\section{Anwendungsbereiche}
Die Anwendung kann in sehr unterschiedlichen Bereichen eingesetzt werden, einige Beispiele sind:
\begin{itemize}
\item Ein Entwickler von Videoencodern kann mit Hilfe von \projektTitel die Qualität der von seinem
		Werkzeug erzeugten Videos auf vielfältige Art und Weise analysieren und die Daten
		entwicklungsunterstützend einsetzen.
\item Ein Internetvideoportal steht vor der Entscheidung welchen Videocodec es verwenden möchte.
		Eine aussagekräftige subjektive Qualitätsbewertung ist mit sehr hohen Kosten verbunden.
		Mit Hilfe von \projektTitel lassen sich diese Kosten vermeiden ohne sich blind für
		einen bestimmten Videocodec zu entscheiden.
\item Ein Forschungsteam des \gls{KIT} möchte die Qualität ihrer Videoencoder analysieren, stellt
		aber fest, dass es für ihren Encoder noch keine aussagekräftigen Analysemetriken gibt.
		Das Forschungsteam möchte aus Zeitgründen keine vollständige Analyseanwendung entwickeln.
		Mit \projektTitel muss das Team lediglich eine passende Analysemetrik als Plugin
		implementieren und kann sich dadurch auf ihr eigentliches Ziel konzentrieren.
\item Entwickler von Filteralgorithmen(z.B. Rauschunterdrückung) können \gls{OQAT} dazu nutzen, Filter (z.b. Rauschen) auf standardisierte Testsequenzen anzuwenden. Ihr eigenes Filterwerkzeug dann auf das daraus entstandene Video anzuwenden und anschließend die Ergebnisse mit Analysemetriken von \projektTitel zu untersuchen.
\end{itemize}
\section{Zielgruppen}
\begin{itemize}
\item Entwickler von Videoencodern.
\item Entwickler von Filteralgorithmen.
\end{itemize}