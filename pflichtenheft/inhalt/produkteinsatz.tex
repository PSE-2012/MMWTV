\chapter{Produkteinsatz}
\gls{OQAT} kann zur Evaluierung von \gls{VBW} eingesetzt werden. 
\section{Anwendungsbereiche}
Die Anwendung kann ihn sehr unterschiedlichen Bereichen eingesetzt werden, einige Beispiele wären:
\begin{itemize}
\item Entwickler von Videoencodern kann mit Hilfe von \projektTitel die Qualität der, von seinem
		Werkzeug, erzeugten Videos auf vielfältiger Art und Weise analysieren und die Daten
		Entwicklungsunterstützend einsetzen.
\item Ein Internetvideoportal steht vor der Entscheidung welchen Videocodec es verwenden möchte.
		Eine aussagekräftige subjektive Qualitätsbewertung ist mit sehr vielen Kosten verbunden.
		Mit Hilfe von \projektTitel lassen sich diese Kosten vermeiden ohne sich blind für
		einen bestimmten Videocodec entscheiden zu müssen.
\item Ein Forschungsteam des \gls{KIT} möchte die Qualität ihrer Videoencoder analysieren, stellt
		aber fest, dass es, für ihren Encoder, noch keine aussagekräftigen Analysemetriken gibt.
		Das Forschungsteam möchte aus Zeitgründen keine vollständige Analyseanwendung entwickeln.
		Kein Problem, das Team braucht lediglich \projektTitel eine neue Analysemetrik nach ihren
		Vorstellungen zu implementieren und kann sich dadurch auf ihr eigentliches Ziel konzentrieren. 
\end{itemize}
\section{Zielgruppen}
\begin{itemize}
\item Entwickler von Videoencodern.
\item Entwickler von Filteralgorithmen(z.B. Rauschunterdrückung) können \gls{OQAT} dazu nutzen Filter (z.b. Rauschen) auf standardisierte Testsequenzen anzuwenden, dann ihr eigenes Filterwerkzeug auf das daraus entstandene Video anwenden und anschließen von einer der, von \gls{OQAT} angebotenen, Analysemetriken zu untersuchen.
\end{itemize}
% % Würde es rausnehmen. Denn es soll bereits in der Produktumgebung angegeben worden sein.
\section{Betriebsbedingungen}
\projektTitel soll auf einem Rechner, wie in der Produktumgebung beschrieben, funktionieren.
Es sollten alle benötigten Komponenten vorhanden sein:
Referenzvideo, Videoverarbeitungswerkzeuge und Optional dazu Parameter für diese.