\chapter{Systemmodelle}
\section{Szenarien}
% Nutzt bitte dieses template für neue Szenarien
% Haltet ein Szenario bitte kürzer als eine Seite (kompilieren-> prüfen), sonst
% läuft es über die Fußzeile hinaus. Man kann es zwar mit longtable umgehen
% aber die Ereignissflussspalte wird dann auf einer eigenen Seite gedruckt.
% Außerdem sollte ein Szenario nicht so lang sein :-).
%\begin{tabular}{p{1.55cm}|p{14cm}}
%Szenario-Name & \\ \hline
%Akteur-Instanzen & \\ \hline
%Ereignis-fluss & \begin{compactenum}[1]
%\item 
%\end{compactenum} \\
%\end{tabular}

\begin{tabular}{p{1.55cm}|p{14cm}}
Szenario-Name & ScharfzeichnerAnalyse\\ \hline
Akteur-Instanzen &Bob:VideoEncoderEntwickler\\ \hline
Ereignis-fluss & \begin{compactenum}[1]
\item Bob startet \projektTitel, es erscheint ein Willkomensfenster auf dem allgemeine Informationen und seine zuletzt geöffneten Projekte angezeigt werden.
\item Bob klickt auf das ">Projekt erstellen"< Symbol aus der Werkzeugleiste.
\item Es öffnet sich ein ">Neues Projekt"< Assistent in dem Bob ein Referenzvideo im \gls{glos:YUV} Format einträgt und angibt, dass das zu testende Werkzeug von ihm manuell aufgerufen wird.
\item Bob überprüft die eingegebenen Informationen und klickt auf den ">Projekt erstellen"< Button. Das neue Projekt wird geöffnet.
\item Bob wählt das von ihm zuvor angegebene Video aus dem Projektexplorer.
\item Bob wählt den Weichzeichner-Filter aus dem Filterkatalog-Tab und klickt auf das ">Filter anwenden"< Symbol aus der Werkzeugleiste 
\item Bob minimiert den \projektTitel, öffnet sein zu testendes Scharfzeichner-Werkzeug und wendet es auf das neu erstellte Video an, dieses befindet sich im Projektordner.
\item Bob wechselt nun wieder zum \projektTitel. Er klickt auf das ">Metriken"<-Tab, anstelle des Filterkatalogs sieht er nun eine Auswahl an Analysemetriken. Bob doppelklickt auf den Eintrag \gls{MSD}.
\item Der Eintrag MSD ist nun hervorgehoben. Bob klickt auf das ">Analyse"< Symbol aus der Werkzeugleiste.
\item Es öffnet sich ein Formular in das Bob den Pfad des Videos angibt das zuvor von seinem Werkzeug erzeugt wurde.
\item Bob klickt auf das ">Analyse"< Symbol aus der Werkzeugleiste
\item Im Log-Fenster wird nun zusätzlich zu den aktuellen Statusmeldungen ein Fortschrittsbalken angezeigt. Bob wartet bis der Vorgang beendet ist.
\item Nach Abschluss des Analysedurchlaufs werden Bob die Ergebnisse der Analyse, im Hauptfenster des \projektTitel, graphisch angezeigt.
\item Bob weiß nun, dass sein Scharfzeichner-Werkzeug den Weichzeichnungseffekt nur mangelhaft beseitigen konnte. Das neu gewonnenen Wissen nutzt er um sein Werkzeug zu verbessern.
\end{compactenum}\\
\end{tabular}

\begin{tabular}{p{1.55cm}|p{14cm}}
Szenario-Name & Fehlende Komponente \\ \hline 
Akteur-Instanzen & \dAU \\ \hline 
Ereignis-fluss & \begin{compactenum}[1]
\item \projektTitel wird von \dAU gestartet, das Wilkommensfenster wird angezeigt.
\item \dAU verwendet ein bestehendes Projekt mit einem bestehendem 
\item Er wählt ein Video aus und klickt das ">Analyse"< Symbol
\item Eine Fehlermeldung erscheint ">Fehlende Komponenten: Werkzeug, Metrik"<
\end{compactenum} \\
\end{tabular}

%-------------------------------------------------------------------------------------------------------
\section{Anwendungsfälle}

% template für Anwendungsfälle, gleiche Vorschläge wie für szenario.
% Beachtet richtlinien für Anwendungsfälle (tichy oder bruegge)
%\begin{tabular}{p{2.2cm}|p{13cm}} \hline
%Anwendungs-fallname & someName\\ \hline
%Akteur & SomeActor\\ \hline
%
%Ereignisfluss & \begin{compactenum}
%\item Some process.
%\end{compactenum}\\ \hline
%
%Anfangs-bedingungen & \begin{compactitem}
%\item Some point.
%\end{compactitem} \\ \hline
%
%Abschluss-bedingungen & \begin{compactitem}
%\item Some point
%\end{compactitem}\\ \hline
%
%Qualitäts-anforderungen & \begin{compactitem}
%\item Some point
%\end{compactitem}\\ \hline
%
%\end{tabular}

%-------------------------------------------------------------------------------------------------------
\section{Objektmodell}
%-------------------------------------------------------------------------------------------------------
\section{Dynamische Modelle}
%-------------------------------------------------------------------------------------------------------
\section{Grafische Benutzerschnittstelle}