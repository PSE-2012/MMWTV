\chapter{Globale Testfälle}
\section{Testfälle}
\setcounter{counterKriterien}{0}

%veraltet => neu schreiben
------------*

\nItem{T} Projekt erstellen, speichern und laden
\nItem{T} Projekt bearbeiten, speichern und laden
\nItem{T} Projekt auf anderem Rechner öffnen

\nItem{T} Filter auswählen ohne Video ausgewählt zu haben - Fehlermeldung
\nItem{T} Filter auswählen, Vorschau betrachten
\nItem{T} Filter anwenden, generierte Video-Datei überprüfen
\nItem{T} Filter-Einstellungen verändern
\nItem{T} Mehrere Filter auf ein Video anwenden, Reihenfolge verändern

\nItem{T} Analysemetrik auswählen ohne zwei Videos ausgewählt zu haben - Fehlermeldung
\nItem{T} Analyse starten ohne Metrik auszuwählen - Fehlermeldung
\nItem{T} Analyse durchführen, Ergebnisse anzeigen
\nItem{T} Analyseergebnisse speichern und laden
\nItem{T} Analyseergebnisse exportieren (csv)

\nItem{T} GUI Funktionalität




\section{Testszenarien}
\setcounter{counterKriterien}{0}

%Projekt erstellen
%Bestehendes Projekt öffnen
%Filter anwenden
%Makrofilter erstellen
%Analyse
%alte Analysedaten anzeigen

\nItem{TS} Neues Projekt und neues Video zum Projekt hinzufügen.\\
\begin{enumerate}
\item \dAU startet \projektTitel.
\item Der Wilkommensbildschirm öffnet sich.
\item \dAU klickt auf den \emph{neues Projekt erstellen} Button.
\item Ein Projekt Erstellungsdialog wird geöffnet.
\item \dAU trägt einen Projekt Titel, eine kurze Beschreibung ein und bestätigt den Vorgang.
\item \dAU klickt auf den Dateiexplorer und wählt \emph{carphone.yuf} mit einem Rechtklick aus. Aus dem erscheinenden Dialog wählt er die \emph{Ressource hinzufügen} Option aus.
\item Das Video wird nun im Projektexplorer angezeigt.
\item \dAU beendet das Programm, da sein Ziel erreicht ist.
\end{enumerate}


\nItem{TS} Bestehendes Projekt öffnen, mehrere Filter anwenden und die Reihenfolge abspeichern.\\ %Filter anwenden/Makrofilter erstellen
\begin{enumerate}
\item \dAU startet \projektTitel
\item Der Willkommesnbildschirm öffnet sich.
\item \dAU wählt aus der Liste seiner zuletzt geöffneten Projekten das \emph{H264Test} Projekt aus.
\item \dAU klickt auf das \emph{carphone.yuf} Video im Smarttree.
\item \dAU wählt den Weichzeichner aus der Filterliste aus.
\item \dAU erhöht die stärke des Weichszeichners und fügt es der Warteliste der Anzuwendenden Filter hinzu.
\item \dAU wählt den Scharfzeichner aus der Filterliste aus und fügt es zur Warteliste der anzuwendenden Filter hinzu. 
\item \dAU schiebt den Scharfzeichner vor den Weichzeichner in der Warteliste.
\item \dAU speichert die Reihenfolge unter dem Namen \emph{ScharfWeich} ab.
\item \dAU klickt auf Filter anwenden.
\item Das Video wird im \emph{smarteTree} als Kindelement des \emph{carphone.YUF} aufgelistet.
\item  \dAU beendet \projektTitel.
\end{enumerate}

\nItem{TS} Analyse durchführen.
\begin{enumerate}
\item \dAU startet \projektTitel.
\item öffnet das \emph{H264Test} Projekt.
\item \dAU wählt im Dateiexplorer die \emph{carphoneH264encoded.YUF} und fügt diese als Kindelement des \emph{carphone.yuv} Video (im emph{smartTree}) hinzu.
\item \dAU wählt die \emph{PSNR} Metrik aus der Metrikenliste aus.
\item \dAU markiert das \emph{carphone.yuv} Video aus dem \emph{smartTree} als Referenzvideo und \emph{carphoneH264encoded.YUF} als das zu analysierende Video.
\item \dAU startet die Analyse durch einen Klick auf das ensprechende Symbol der Toolbar.
\item \dAU wartet bis der Analysevorgang beendet wurde.
\item \dAU schaut sich die Ergebnisse der Analyse an und gibt eine Beschreibung an die er zusammen mit den Analyseergebnissen abspeichert um auch später auf diese zurückgreifen zu können.
\item \dAU schließt das Programm.
\end{enumerate}

\nItem{TS} 