\chapter{Globale Testfälle}
\section{Testfälle}
\setcounter{enumi}{0}
\nItem{T} Analysedaten sollen speichern/laden.
\nItem{T} Gui Funktionalität testen.
\nItem{T} Videos sollen korrekt im Projekt abgelegt werden.
\nItem{T} Analysedaten visuell darstellen.
\nItem{T} Projekt auf anderen Rechnern öffnen.
\nItem{T} Filter und Metriken sollen korrekt auf Videos angewand werden

\section{Testszenarien}
\nItem{T}  \begin{enumerate}
\item Der Benutzer startet das Programm.
\item Das Programm begrüßt den Nutzer mit dem Willkommensbildschirm. Da noch keine Projekte vorhanden sind, sind die meisten Buttons grau deaktiviert und grau hinterlegt.
\item Der Nutzer klickt auf den ">Projekt erstellen"< Button.
\item Es erscheint ein Fenster, welche einen Projektnamen verlangt sowie ein Testvideo. Optional lässt sich eine Beschreibung eintragen.
\item Der Nutzer gibt einen Namen und eine Beschreibung ein und läd über den im Programm eingefügten Explorer ein Originalvideo in das Projekt.
\item Das Programm erstellt das Projekt und stellt die Ordnerstruktur bereit.
\item Der Nutzer beendet das Programm.

\end{enumerate}

\nItem{T}  \begin{enumerate}
\item Der Benutzer startet das Programm.
\item Das Programm begrüßt den Nutzer mit dem Willkommensbildschirm.
\item Der Nutzer wählt mit einem Linksklick im Linken Explorer Fenster ein Projekt.
\item Das Programm öffnet die Ordnerstruktur und zeigt im linken Explorerfenster nun alle Dateien des Projektes an.
\item Der Nutzer wählt im rechten Werkzeugsfenster den Flimmer Filter und den Weichzeichner. Dazu gibt er dem Programm noch den Grad der beiden Einstelungen an.
\item Er bestätigt und klickt auf den ">Testviedeo erstellen"< Button.
\item Die Software fängt an das Video zu erstellen und gib in einem Pop-up fenster den aktuellen Fortschritt an. Nachdem das video erstellt ist verschwindet das pop up und die software signalisiert ihren Fortschritt mit einem Ton.
\item Der Nutzer beendet das Programm.
\end{enumerate}