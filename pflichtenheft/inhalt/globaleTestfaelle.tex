\chapter{Globale Testfälle}
\section{Testfälle}
\setcounter{counterKriterien}{0}

\nItem{T} Projekt erstellen, speichern und laden
\nItem{T} Projekt bearbeiten, speichern und laden
\nItem{T} Projekt auf anderem Rechner öffnen

\nItem{T} Filter auswählen ohne Video ausgewählt zu haben - Fehlermeldung
\nItem{T} Filter auswählen, Vorschau betrachten
\nItem{T} Filter anwenden, generierte Video-Datei überprüfen
\nItem{T} Filter-Einstellungen verändern
\nItem{T} Mehrere Filter auf ein Video anwenden, Reihenfolge verändern

\nItem{T} Analysemetrik auswählen ohne zwei Videos ausgewählt zu haben - Fehlermeldung
\nItem{T} Analyse starten ohne Metrik auszuwählen - Fehlermeldung
\nItem{T} Analyse durchführen, Ergebnisse anzeigen
\nItem{T} Analyseergebnisse speichern und laden
\nItem{T} Analyseergebnisse exportieren (csv)

\nItem{T} GUI Funktionalität




\section{Testszenarien}
\setcounter{counterKriterien}{0}
\nItem{TS}   \begin{compactenum}[1]
\item Der Benutzer startet das Programm.
\item Das Programm begrüßt den Nutzer mit dem Willkommensbildschirm. Da noch keine Projekte vorhanden sind, sind die meisten Buttons deaktiviert.
\item Der Nutzer klickt auf den ">Projekt erstellen"< Button.
\item Es erscheint ein Fenster, welches einen Projektnamen verlangt sowie ein Testvideo. Optional lässt sich eine Beschreibung eintragen.
\item Der Nutzer gibt einen Namen und eine Beschreibung ein und lädt über den im Programm eingefügten Explorer ein Originalvideo in das Projekt.
\item Das Programm erstellt das Projekt und stellt die Ordnerstruktur bereit.
\item Der Nutzer beendet das Programm.

\end{compactenum}

\nItem{TS}  \begin{compactenum}[1]
\item Der Benutzer startet das Programm.
\item Das Programm begrüßt den Nutzer mit dem Willkommensbildschirm.
\item Der Nutzer wählt mit einem Linksklick im linken Explorerfenster ein Projekt.
\item Das Programm öffnet die Ordnerstruktur und zeigt im linken Explorerfenster nun alle Dateien des Projektes an.
\item Der Nutzer wählt im rechten Werkzeugsfenster den Flimmer Filter und den Weichzeichner. Dazu gibt er dem Programm noch den Grad der beiden Einstellungen an.
\item Er bestätigt und klickt auf den ">Testviedeo erstellen"< Button.
\item Die Software fängt an das Video zu erstellen und zeigt den aktuellen Fortschritt an. Nachdem das Video erstellt ist signalisiert \projektTitel die Fertigstellung mit einem Ton.
\item Der Nutzer beendet das Programm.
\end{compactenum}

\nItem{TS}
\begin{compactenum}[1]
\item \projektTitel wird von \dAU gestartet, das Willkommensfenster wird angezeigt.
\item \dAU verwendet ein bestehendes Projekt.
\item Er wählt ein Video aus und klickt das ">Analyse"< Symbol
\item Eine Fehlermeldung erscheint ">Fehlende Komponenten: Werkzeug, Metrik"<
\end{compactenum}

\nItem{TS} \begin{compactenum}[1]
\item \dAU  startet \projektTitel, es erscheint ein Willkomensfenster auf dem allgemeine Informationen und seine zuletzt geöffneten Projekte angezeigt werden.
\item \dAU  klickt auf das ">Projekt erstellen"< Symbol aus der Werkzeugleiste.
\item Es öffnet sich ein ">Neues Projekt"< Assistent, in dem Bob ein Referenzvideo im \gls{glos:YUV} Format einträgt.
\item \dAU  überprüft die eingegebenen Informationen und klickt auf den ">Projekt erstellen"< Button. Das neue Projekt wird geöffnet.
\item \dAU  wählt das von ihm zuvor angegebene Video aus dem Projektexplorer.
\item \dAU  wählt den Weichzeichner-Filter aus dem Filterkatalog-Tab. Im Filter-Paramter-Abschnitt stellt er die Intensität des Weichzeichner-Effekts ein und klickt anschließend auf das ">Filter anwenden"< Symbol aus der Werkzeugleiste 
\item \dAU  minimiert nun \projektTitel und öffnet sein zu testendes Scharfzeichner-Werkzeug auf dem gerade durch den Weichzeichner-Filter generierten Testvideo, das neu im Projektordner erstellt wurde.
\item \dAU  wechselt nun wieder zu \projektTitel. Er wählt in der Werkzeugleiste die Option ">Testvideo hinzufügen"< und lädt das gerade durch sein Scharfzeichner-Werkzeug erstellte Video in sein Projekt.
\item \dAU  wählt das ursprüngliche Referenzvideo ohne Weichzeichner-Effekt und sein gerade vom Scharfzeichner bearbeitetes Video an, um diese zu vergleichen.
\item Dazu klickt \dAU  auf das ">Metriken"<-Tab. Anstelle des Filterkatalogs sieht er nun eine Auswahl an Analysemetriken. Bob doppelklickt auf den Eintrag \gls{MSE}.
\item Der Eintrag \gls{MSE} ist nun hervorgehoben. \dAU  klickt auf das ">Analyse"< Symbol aus der Werkzeugleiste.
\item Im Log-Fenster wird nun zusätzlich zu den aktuellen Statusmeldungen ein Fortschrittsbalken angezeigt. Bob wartet bis der Vorgang beendet ist.
\item Nach Abschluss des Analysedurchlaufs werden \dAU  die Ergebnisse der Analyse im Hauptfenster von \projektTitel graphisch dargestellt.
\item \dAU  weiß nun, dass sein Scharfzeichner-Werkzeug den Weichzeichnungseffekt nur mangelhaft beseitigen konnte. Das neu gewonnenen Wissen nutzt er um sein Werkzeug zu verbessern.
\end{compactenum}
