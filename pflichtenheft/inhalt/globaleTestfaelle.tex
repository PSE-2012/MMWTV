\chapter{Globale Testfälle}
\section{Testfälle}
\setcounter{counterKriterien}{0}

%veraltet => neu schreiben
------------*

\nItem{T} Projekt erstellen, speichern und laden
\nItem{T} Projekt bearbeiten, speichern und laden
\nItem{T} Projekt auf anderem Rechner öffnen

\nItem{T} Filter auswählen ohne Video ausgewählt zu haben - Fehlermeldung
\nItem{T} Filter auswählen, Vorschau betrachten
\nItem{T} Filter anwenden, generierte Video-Datei überprüfen
\nItem{T} Filter-Einstellungen verändern
\nItem{T} Mehrere Filter auf ein Video anwenden, Reihenfolge verändern

\nItem{T} Analysemetrik auswählen ohne zwei Videos ausgewählt zu haben - Fehlermeldung
\nItem{T} Analyse starten ohne Metrik auszuwählen - Fehlermeldung
\nItem{T} Analyse durchführen, Ergebnisse anzeigen
\nItem{T} Analyseergebnisse speichern und laden
\nItem{T} Analyseergebnisse exportieren (CSV)

\nItem{T} GUI Funktionalität




\section{Testszenarien}
\setcounter{counterKriterien}{0}

%Projekt erstellen
%Bestehendes Projekt öffnen
%Filter anwenden
%Makrofilter erstellen
%Analyse
%alte Analysedaten anzeigen

\nItem{TS} Neues Projekt und neues Video zum Projekt hinzufügen.\\
\begin{enumerate}
\item \dAU startet \projektTitel.
\item Der Wilkommensbildschirm öffnet sich.
\item \dAU klickt auf den \emph{Neues Projekt erstellen} Button.
\item Ein Projekt-Erstellungsdialog wird geöffnet.
\item \dAU trägt einen Projekt-Titel und eine kurze Beschreibung ein und bestätigt den Vorgang.
\item \dAU klickt auf den Dateiexplorer und wählt \emph{carphone.yuf} mit einem Rechtklick aus. Aus dem erscheinenden Dialog wählt er die \emph{Ressource hinzufügen} Option.
\item Das Video wird nun im Projektexplorer angezeigt.
\item \dAU beendet das Programm, da sein Ziel erreicht ist.
\end{enumerate}


\nItem{TS} Bestehendes Projekt öffnen, mehrere Filter anwenden und deren Reihenfolge abspeichern.\\ %Filter anwenden/Makrofilter erstellen
\begin{enumerate}
\item \dAU startet \projektTitel
\item Der Willkommesnbildschirm öffnet sich.
\item \dAU wählt aus der Liste seiner zuletzt geöffneten Projekten das \emph{H264Test} Projekt aus.
\item \dAU klickt auf das \emph{carphone.yuf} Video im Smarttree.
\item \dAU wählt den Weichzeichner aus der Filterliste aus.
\item \dAU erhöht die Stärke des Weichszeichners und fügt ihn der Warteliste anzuwendender Filter hinzu.
\item \dAU wählt den Scharfzeichner aus der Filterliste aus und fügt ihn der Warteliste anzuwendender Filter hinzu.
\item \dAU schiebt den Scharfzeichner vor den Weichzeichner in der Warteliste.
\item \dAU speichert die Reihenfolge unter dem Namen \emph{ScharfWeich} ab.
\item \dAU klickt auf Filter anwenden.
\item Das Video wird im \emph{SmartTree} als Kindelement von \emph{carphone.YUF} aufgelistet.
\item  \dAU beendet \projektTitel.
\end{enumerate}

\nItem{TS} Analyse durchführen.
\begin{enumerate}
\item \dAU startet \projektTitel.
\item \dAU öffnet das \emph{H264Test} Projekt.
\item \dAU wählt im Dateiexplorer die \emph{carphoneH264encoded.YUF} und fügt diese als Kindelement des \emph{carphone.yuv} Video (im emph{smartTree}) hinzu.
\item \dAU wählt die \emph{PSNR} Metrik aus der MetriList aus.
\item \dAU markiert das \emph{carphone.yuv} Video aus dem \emph{SmartTree} als Referenzvideo und \emph{carphoneH264encoded.YUF} als das zu analysierende Video.
\item \dAU startet die Analyse durch einen Klick auf das ensprechende Symbol der Toolbar.
\item \dAU wartet bis der Analysevorgang beendet wurde.
\item \dAU schaut sich die Ergebnisse der Analyse an und gibt eine Beschreibung ein, die er zusammen mit den Analyseergebnissen abspeichert um auch später auf diese zurückgreifen zu können.
\item \dAU schließt das Programm.
\end{enumerate}

\nItem{TS} Kompletter Analyse-Ablauf
\begin{enumerate}
\item \dAU startet \projektTitel.
\item Es erscheint ein Wilkommensbildschirm.
\item \dAU wählt die Option \emph{neues Projekt erstellen}.
\item Im geöfneten Projekterstellungsdialog trägt \dAU den Projekttitel (mpeg-4-TestProjekt) und den Pfad(~/bob/Videos/) zu einem Videoverzeichnis in seinem Arbeitsverzeichnis.
\item \dAU klickt auf den \emph{Projekt anlegen} Button.
\item Im \emph{smartTree} des \emph{ProjektExplorers} werden nun alle kompatiblen (.YUV) Videos, die sich
im Videoverzeichnis von \dAU befinden, aufgelistet.
\item \dAU wählt den \emph{foreman.yuv} Eintrag im Smarttree aus.
\item \dAU klickt auf das \emph{play} Symbol des Videoplayers (im Hauptvisualisierungsbereich).
\item Nachdem \dAU sich vergewissert hat, dass die Videosequenz seinen Ansprüchen genügt pausiert er
das Video (durch einen Klick auf das \emph{pause} Symbol des Videoplayers).
\item \dAU wählt einen Rauschfilter aus der Filterliste.
\item \dAU wählt den \emph{Weichzeichner} Filter aus der Filterliste aus und fügt ihn der Warteschlange der auf das Video anzuwendenden Filter hinzu.
\item \dAU gibt einen Namen an (foremanRauschenWeich.yuv), klickt auf den \emph{Filter anwenden} Button und wartet bis der Vorgang beendet wurde.
\item Im SmartTree wird nun \emph{foremanRauschenWeich.yuv} als Kindelement von \emph{foreman.yuv} aufgeführt(und befindet sich außerdem im Projektordner von Bob).
\item Bob startet seinen MPEG-4 Encoder, wendet ihn auf das \emph{foremanRauschenWeich.yuv} Video aus seinem Projektverzeichnis an und speichert das erzeugte Video (\emph{foremanRauschenWeichEncoded.yuv} Video in seinem Projektverzeichnis ab.
\item Bob wählt das \emph{foremanRauschenWeichEncoded.yuv} Video aus dem Dateiexplorer( ein Tab im Projektexplorer) und fügt es als Kindelement von \emph{foremanRauschenWeich.yuv} hinzu.
\item Bob wählt den Analyse-Tab (neben dem Filter-Tab) aus, im Videoplayer sind nun zwei leere Boxen zu sehen.
\item Bob wählt das \emph{foreman.yuv} Video als Referenzvideo und das \emph{foremanRauschenWeichEncoded.yuv} als das zu vergleichende Video aus. In der linken (zuvor leeren) Box wird nun das Referenzvideo und in der rechten Box das zu vergleichende Video angezeigt.
\item Bob wählt die PSNR Metrik aus der Metriken Liste aus und klickt auf den \emph{Analyse starten} Button.
\item Bob wartet bis der Analysevorgang fertig ist und startet das Ergebnisvideo (im Player).
\item Nachdem Bob sich das Ergebnisvideo angeschaut hat schreibt er eine kurze Beschreibung der Analyseergebnisse und speichert diese ab.
\item Bob schließt \projektTitel.
\end{enumerate}
