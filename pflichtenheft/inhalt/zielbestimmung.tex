\chapter{Zielbestimmung}

\section{Mußkriterien}

\newItemMK 1 Nach einem Analysedurchlauf soll dem Nutzer die Möglichkeit angeboten werden, die entstandenen Daten zu exportieren.
\newItemMK 2 Ein gegeben Video im YUV Format kann mit verschiedenen Testsignalen verfälscht werden
\newItemMK 3 Die Anwendung sollte ohne großen Aufwand um neue Filter oder Bewertungsmetriken erweitert werden können.
\newItemMK 4 Die Ergebnisse eines Analysedurchlaufs sollen Graphisch aufbereitet werden.
\newItemMK 5  Viele Standarttestsignale/Filter/Bewertungsmetriken sollen der Anwendung bereits zum Auslieferungszeitpunkt zur Verfügung stehen
\newItemMK 6 Die Anwendung soll auf einem im Handel erwerbbaren Rechner lauffähig sein.

\newItemMK 7 Benutzer haben die Möglichkeiten Projekte anzulegen

\newItemMK 8 Die Anwendung kann Videos im .YUV Format von der Festplatte laden, manipulieren und wieder auf die Festplatte schreiben.

\newItemMK 9 

\newItemMK 10 
\setcounter{enumi}{0}

\section{Wunschkriterien}

\newItemWK 1 Die Daten die der InstitutsEncoder bereitstellt sollen visuell aufbereitet werden
\newItemWK 2 Der Endnutzer sollte ohne aufwendiger Einarbeitung das Programm bedienen können.
\newItemWK 3 Analyseergebnisse können zu einem bestimmten Videoverarbeitungswerkzeug zugeordnet und chronologisch abgespeichert werden.
\newItemWK 4 Das Programm ist in der Lage .YUV Video mit verschiedenen Testsignalen zu verschmelzen.
\newItemWK 5 Der Nutzer hat die Möglichkeit die Ergebnisse eines Analysedurchlaufs zu exportieren.
\newItemWK 6 Neue Filter/Metriken können auch nach Auslieferung des Programms zur Anwendung hinzugefügt werden. Dazu
  sind keine Änderungen am der Anwendung selbst notwendig
\newItemWK 7 Das nachträgliche portieren der Anwendung in eine andere Sprache sollte möglichst einfach sein.

\newItemWK 8 Der Nutzer soll die möglichkeit haben seine erstellten Projekte auf jedem anderen Computer nutzen zu können.

\setcounter{enumi}{0}

\section{Abgrenzungskriterien}

\newItemAK 1 Audiounterstützung ist nicht geplant
\newItemAK 2 keine Analysierung der Leistungsdaten
\newItemAK 3 Da die Videoverarbeitungswerkzeuge nicht direkt in die zu entwickelnde Anwendung eingebunden werden sind keine Echtzeitanalysen vorgesehen.


\setcounter{enumi}{0}
