\chapter{Zielbestimmung}

\section{Mußkriterien}

\newItemMK Nach einem Analysedurchlauf soll dem Nutzer die Möglichkeit angeboten werden, die entstandenen Daten zu exportieren. %1
\newItemMK Ein gegebenes Video im YUV Format kann mit verschiedenen Testsignalen verfälscht werden %2
\newItemMK Die Anwendung sollte ohne großen Aufwand um neue Filter oder Bewertungsmetriken erweitert werden können. %3
\newItemMK Die Ergebnisse eines Analysedurchlaufs sollen Graphisch aufbereitet werden. %4
\newItemMK Ein bestimmter Satz an Filtern und Analysemetriken soll der Anwendung bereits zum Auslieferungszeitpunkt zur Verfügung stehen.
\newItemMK Benutzer haben die Möglichkeiten Projekte anzulegen, solch einem Projekt diehnt als Ablageort für Daten über 
\begin{itemize}
\item absolvierte Analysevorgänge
\item relevante Projekteinstellungen
\item vom Benutzer bereitgestellte Resourcen (z.B. ein Video)
\end{itemize}
\newItemMK Die Anwendung kann Videos im \gls{glos:YUV} Format von der Festplatte lesen, manipulieren und wieder auf die Festplatte schreiben.


\setcounter{enumi}{0}

\section{Wunschkriterien}

\newItemWK Die Daten die der der \gls{glos:VE} vom \gls{ITEC}  bereitstellt sollen visuell aufbereitet werden
\newItemWK Der Nutzer sollte ohne langer Einarbeitungsphase das Programm bedienen können.

\newItemWK Das nachträgliche portieren der Anwendung in eine andere Sprache sollte möglichst einfach sein.

\newItemWK Ein angelegtes Projekt ist auch unter einer anderen \projektTitel Installation lauffähig, vorausgesetzt die nötigen Resourcen (z.B. Video Dateien die dem Projekt zuvor zur Verfügung gestellt wurden) stehen bereit.

\setcounter{enumi}{0}

\section{Abgrenzungskriterien}

\newItemAK Audiounterstützung ist nicht geplant
\newItemAK \projektTitel verfolgt nicht das Ziel ein Videoverarbeitungswerkzeug auf seine Performanz zu untersuchen.
\newItemAK Da die Videoverarbeitungswerkzeuge nicht direkt in die zu entwickelnde Anwendung eingebunden werden sind keine Echtzeitanalysen vorgesehen.


\setcounter{enumi}{0}
