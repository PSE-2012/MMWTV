\chapter{Zielbestimmung}

\section{Musskriterien}
\setcounter{counterKriterien}{0}
\nItem{MK} \projektTitel verfügt über eine interaktive graphische Oberfläche.\\
\nItem{MK} Die Anwendung kann Videos im \gls{glos:YUV} Format von der Festplatte lesen, manipulieren und wieder auf die Festplatte schreiben.\\
\nItem{MK} \projektTitel kann ein bereitgestelltes Video darstellen und bietet grundlegende Kontrollfunktionalitäten an wie:
\begin{compactitem}
\item zeige das nächste Frame
\item zeige das vorherige Frame
\item zeige ein vom Nutzer gewähltes Frame
%\item spiele das Video mit einer Geschwindigkeit von X \footnote{ natürlich ist |X| nach oben beschränkt. Die tatsächliche Beschränkung von X wird im Laufe der Entwicklung festgelegt aber es wird angenommen das X nicht viel größer 100 FPS sein darf.} \gls{fps}.
\item spiele das Video mit einer Variablen Geschwindigkeit ab
\item halte das Video an	
\end{compactitem} 
\nItem{MK} Ein Satz an Filtern steht der Anwendung bereits zum Auslieferungszeitpunkt zur Verfügung. 
Diese sind:
\begin{compactitem}
\item Rauschgenerator
\item Verschiedene Farbfilter (Graustufen, Sepia, Farbkanäle ein/ ausblenden)
\item Faltungs-Filter (Scharfzeichner, Weichzeichner, Sobel-Filter)
%\item Posterisation
\end{compactitem}
\nItem{MK} Ein Satz an Analysemetriken steht der Anwendung bereits zum Auslieferungszeitpunkt zur Verfügung.
Diese sind:
\begin{compactitem}
\item Mean square error \\
Der Mean Square Error dient zur Berechnung der Abweichung eines Schätzers von dem zu schätzenden Wert. Es ergibt sich aus der Differenz zwischen den Werten und dem arithmetischem Mittel. Die Differenz wird quadriert und durch die Anzahl der Werte dividiert.
\item Signal-to-noise ratio \\
Das Signal-to-noise Ratio ist ein Maß für die Qualität eines Signals, das von einem Rauschsignal überlagert ist. Es ist definiert als das Verhältnis der mittleren Leistung des Signals zur mittleren Rauschleistung des Störsignals.
\item Peak signal-to-noise ratio
PSNR ist ein Maßstab zur Messung von Unterschieden zwischen zwei Bildsignalen. Er wird unter anderem dazu verwendet, um mittels eines unkomprimierten Original-Videos und eines verlustbehaftet komprimierten entsprechenden Vergleichs-Videos den Qualitätsverlust zu quantifizieren. Der PSNR basiert auf der MSE für die Pixel des Original-Videos und des Vergleichsvideos und wird durch eine Formel errechnet. Bei Farbsignalen wird der PSNR separat je Farbkomponente errechnet und anschliessend der Mittelwert gebildet.
% % hier sollten noch ein paar rein, vlcht SSIM ?
\end{compactitem}
\nItem{MK} Falls es für einen Filter sinnvolle Einstellungsmöglichkeiten gibt, kann man diese verändern. Beispielsweise kann die Stärke eines Weichzeichners eingestellt werden.\\
\nItem{MK} Einstellungsparameter für Filter können abgespeichert und geladen werden.\\
\nItem{MK} Ein Filtervorgang kann mehrere Filter beinhalten. Die Reihenfolge, in der Filter angewandt
werden, ist variabel und kann vom Nutzer bestimmt und abgespeichert werden. Bei einem solchen Vorgang wird nur ein Video für den kompletten Filtervorgang abgespeichert.\\
\nItem{MK} Benutzer haben die Möglichkeiten Projekte anzulegen. Ein Projekt dient als Ablageort für
absolvierte Analysevorgänge, relevante Projekteinstellungen und die vom Benutzer bereitgestellte Ressourcen.\\
\nItem{MK} Der Nutzer hat die Möglichkeit aktuelle und oder zurückliegende Analyseergebnisse in einem portablen Format (z.B. \gls{csv}) zu exportieren.\\
\nItem{MK} Die Ergebnisse eines Analysedurchlaufs werden graphisch aufbereitet (z.B. als Differenzbild).\\
\nItem{MK} \projektTitel unterstützt Lokalisierung, d.h. der Nutzer hat die Möglichkeit die Sprachfunktionalität der Anwendung durch Sprachpakete zu erweitern. Es wird nur die deutsche Sprachdatei standardmäßig bereitgestellt.

\section{Wunschkriterien}
\setcounter{counterKriterien}{0}
% 
%\nItem{WK} Es wird eine Vorschau, basierend auf den ausgewählten Filtern, generiert.
\nItem{WK} \projektTitel kann die Daten, die der \gls{glos:VE} des \gls{ITEC} bereitstellt, erkennen und darstellen.\\
\nItem{WK} Ein angelegtes Projekt ist auch unter einer anderen \projektTitel Installation lauffähig, vorausgesetzt die nötigen Resourcen (z.B. Video Dateien die dem Projekt zuvor zur Verfügung gestellt wurden)  sind vorhanden und es handelt sich um die selbe Version von \projektTitel.\\
% Metriken als plugins einbinden zu können wird eine ziemlich harte Nuss...
% Filter sind da nicht so schwierig denke ich..
\nItem{WK} Der Benutzer kann eigene Analysemetriken implementieren und diese als Plugins einbinden.\\
\nItem{WK} Der Benutzer kann eigene Filter implementieren und diese als Plugins einbinden.\\
\nItem{WK} Während eines Analysevorgangs wird eine Fortschrittsanzeige und die voraussichtliche Dauer des Vorgangs angezeigt.\\
% Ein beinahe identisches MK befindet sich ein paar Zeilen weiter oben.
%\nItem{WK}	Während eines Analysevorgangs werden die vom \projektTitel durchgeführten Operation in einem Statusfenster dargestellt und optional in einem Logfile gespeichert.
\nItem{WK} Für einen ausgewählten Filter wird im Visualisierungsbereich eine Vorschau generiert.\\
\nItem{WK} Der Projektexplorer unterstützt Drag and Drop.
\section{Abgrenzungskriterien}
\setcounter{counterKriterien}{0}
\nItem{AK} \projektTitel beherrscht nur \gls{glos:FR} Analysemetriken.\\
\nItem{AK} \gls{OQAT} arbeitet nicht mit Audiodaten.\\
\nItem{AK} \projektTitel untersucht nicht die Performanz von \gls{glos:VBW}en\\
