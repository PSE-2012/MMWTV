\chapter{Zielbestimmung}

\section{Mußkriterien}
\setcounter{counterKriterien}{0}
\nItem{MK} Die Anwendung kann Videos im .YUV Format von der Festplatte lesen, manipulieren und wieder auf die Festplatte schreiben.
\nItem{MK} Ein bestimmter Satz an Filtern und Analysemetriken soll der Anwendung bereits zum Auslieferungszeitpunkt zur Verfügung stehen.
\nItem{MK} Benutzer haben die Möglichkeiten Projekte anzulegen, solch ein Projekt dient als Ablageort für Daten über:
\begin{compactitem}
\item absolvierte Analysevorgänge
\item relevante Projekteinstellungen
\item vom Benutzer bereitgestellte Resourcen (z.B. ein Video)
\end{compactitem}
\nItem{MK} Nach einem Analysedurchlauf soll dem Nutzer die Möglichkeit angeboten werden, die entstandenen Daten zu exportieren.
\nItem{MK} Nach einem Analysedurchlauf wird eine Bewertung erstellt.

\section{Wunschkriterien}
\setcounter{counterKriterien}{0}
\subsection{Graphische Bedienoberfläche}
\nItem{WK} Es wird eine Vorschau, basierend auf den ausgewählten Filtern, generiert.
\nItem{WK} Die Ergebnisse eines Analysedurchlaufs sollen graphisch aufbereitet werden.
\nItem{WK} Die Daten die der \gls{glos:VE} vom \gls{ITEC}  bereitstellt sollen visuell aufbereitet werden.
\subsection{Portabilität}
\nItem{WK} Ein angelegtes Projekt ist auch unter einer anderen \projektTitel Installation lauffähig, vorausgesetzt die nötigen Resourcen (z.B. Video Dateien die dem Projekt zuvor zur Verfügung gestellt wurden) stehen bereit und es handelt sich um die selbe VErsion von \projektTitel.
\nItem{WK} Es sollen verschiedene Videoverarbeitungswerkzeuge benutzt werden können.
\nItem{WK} Für \projektTitel ist es unerheblich welches \gls{VBW} verwendet wurde, solange es in einem
			kompatiblem Format zur Verfügung gestellt wird.
\nItem{WK} Der Nutzer hat die Möglichkeit aktuelle und/oder zurückliegende Analyseergebnisse in einem
			portablem Format (z.B. PDF) zu exportieren.
\nItem{WK} \projektTitel kann auch mit anderen verbreiteten (aber Verlustfreien) Video Codecs umgehen.
\subsection{Analyse}
\nItem{WK} Der Benutzer kann eigene Bewertungsmetriken definieren, dafür sollten ihm einige Beispiel
			Metriken zur Verfügung gestellt werden.
\nItem{WK} Während eines Analysevorgangs wird eine Fortschrittsanzeige und die voraussichtliche Dauer
			des Vorgangs angezeigt.
\nItem{WK}	Während eines Analysevorgangs wird der Grobablauf in dem, von der Bedienoberfläche zur
			 Verfügung gestelltem, Statusfenster dokumentiert.
\nItem{WK} 	Analysedaten werden mit einem Projekt verknüpft, so kann der Benutzer auf diese später
			zurückgreifen um Sie mit z.B. aktuellen Daten zu vergleichen.
\subsection{Verzerrung}
\nItem{WK} Der Benutzer kann eigene Filter definieren, dafür sollten ihm einige Beispiel Filter zur
			Verfügung gestellt werden.
\nItem{WK} Es ist möglich eigene Videoverarbeitungswerkzeuge als Filter einzubinden, vorausgesetzt
			das einzubindende Werkzeug besitzt eine Kommandozeilenschnittstelle.
\nItem{WK} Wenn ein Benutzer sein \gls{VBW} in \projektTitel einbindet(als externen Filter) hat er die
			Möglichkeit die von diesem \gls{VBW} akzeptierten Parameter und ihre Typen{int, String}
			festzulegen. Diese Parameter können dann von der Bedienoberfläche verwaltet werden und, 
			kommandozeilenbasiert, an das entsprechende Werkzeug beim Aufruf übergeben werden.
\nItem{WK} Es ist möglich Einstellungen für einen bestimmten Filter(wobei es auch ein externer sein kann)
			 als Favoriten abzuspeichern um auf diese später zurückgreifen zu können.
\nItem{WK} Für bekannte Filter werden interaktive Werkzeuge angeboten, die es erlauben einige Parameter
			manuell und in Echtzeit festzulegen. Z.b. Wird eine Schiebeleiste angeboten mit der
			man den Grad des Weichzeichnerfilters einstellen kann und die Auswirkungen in Echtzeit auf
			dem zu verzerrendem Video sieht.
\nItem{WK} \projektTitel erlaubt eine Einstellung der Reihenfolge in der Filter angewandt werden. Solch
			eine Reihenfolge darf als Favorit gespeichert werden um auf diese später zurückgreifen zu können.
\nItem{WK}	Während eines Verzerrungsvorgangs wird der Grobablauf in dem, von der Bedienoberfläche zur
			 Verfügung gestellten, Statusfenster dokumentiert.
\nItem{WK}	Es ist möglich einige generelle Einstellungen am Video vorzunehmen (z.b. Farbkanälle 
			nach Bedarf zu deaktivieren).

\section{Abgrenzungskriterien}
\setcounter{counterKriterien}{0}
\nItem{AK} Audiounterstützung ist nicht geplant
\nItem{AK} \projektTitel verfolgt nicht das Ziel ein Videoverarbeitungswerkzeug auf seine Performanz zu untersuchen.
\nItem{AK} Da die \gls{VBW} nicht direkt in \projektTitel eingebunden
			werden, kann \projektTitel nicht auf auf Echtzeitdaten eines \gls{VBW} zugreifen.

