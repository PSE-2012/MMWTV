\chapter{Zielbestimmung}

\section{Mußkriterien}
\setcounter{counterKriterien}{0}
 
\nItem{MK} Die Anwendung kann Videos im .YUV Format von der Festplatte lesen, manipulieren und wieder auf die Festplatte schreiben.
\nItem{MK} Ein bestimmter Satz an Filtern und Analysemetriken soll der Anwendung bereits zum Auslieferungszeitpunkt zur Verfügung stehen.
\nItem{MK} Benutzer haben die Möglichkeiten Projekte anzulegen, solch ein Projekt dient als Ablageort für Daten über:
\begin{compactitem}
\item absolvierte Analysevorgänge
\item relevante Projekteinstellungen
\item vom Benutzer bereitgestellte Resourcen (z.B. ein Video)
\end{compactitem}
\nItem{MK} Nach einem Analysedurchlauf soll dem Nutzer die Möglichkeit angeboten werden, die entstandenen Daten zu exportieren.
\nItem{MK} Nach einem Analysedurchlauf wird eine Bewertung erstellt.

\section{Wunschkriterien}
\setcounter{counterKriterien}{0}
\nItem{WK} Die Anwendung sollte ohne großen Aufwand um neue Filter oder Bewertungsmetriken erweitert werden können.
\nItem{WK} Die Ergebnisse eines Analysedurchlaufs sollen graphisch aufbereitet werden.
\nItem{WK} Die Daten die der \gls{glos:VE} vom \gls{ITEC}  bereitstellt sollen visuell aufbereitet werden
\nItem{WK} Ein angelegtes Projekt ist auch unter einer anderen MMWTV Installation lauffähig, vorausgesetzt die nötigen Resourcen (z.B. Video Dateien die dem Projekt zuvor zur Verfügung gestellt wurden) stehen bereit.
\nItem{WK} Es sollen verschiedene Videoverarbeitungswerkzeuge benutzt werden können.
\nItem{WK} Das .YUV Format kann in ein anderes Format umgewandelt werden.
\nItem{WK} Es sollen Optional Parameter für Videoverarbeitungswerkzeuge eingegeben werden können.
\nItem{WK} Analysedurchlaufs ein Fortschrittsbalken und Statusmeldungen angezeigt.
\nItem{WK} Ein Vergleich der Analysedaten soll möglich sein.

\section{Abgrenzungskriterien}
\setcounter{counterKriterien}{0}
\nItem{AK} Audiounterstützung ist nicht geplant
\nItem{AK} \projektTitel verfolgt nicht das Ziel ein Videoverarbeitungswerkzeug auf seine Performanz zu untersuchen.
\nItem{AK} Da die Videoverarbeitungswerkzeuge nicht direkt in die zu entwickelnde Anwendung eingebunden werden sind keine Echtzeitanalysen vorgesehen.

