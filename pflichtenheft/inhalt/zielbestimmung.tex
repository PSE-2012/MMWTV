\chapter{Zielbestimmung}

\section{Mußkriterien}
\setcounter{enumi}{0}
\nItem{MK} Nach einem Analysedurchlauf soll dem Nutzer die Möglichkeit angeboten werden, die entstandenen Daten zu exportieren. %1
\nItem{MK} Ein gegebenes Video im YUV Format kann mit verschiedenen Testsignalen verfälscht werden %2
\nItem{MK} Die Anwendung sollte ohne großen Aufwand um neue Filter oder Bewertungsmetriken erweitert werden können. %3
\nItem{MK} Die Ergebnisse eines Analysedurchlaufs sollen Graphisch aufbereitet werden. %4
\nItem{MK} Ein bestimmter Satz an Filtern und Analysemetriken soll der Anwendung bereits zum Auslieferungszeitpunkt zur Verfügung stehen.
\nItem{MK} Benutzer haben die Möglichkeiten Projekte anzulegen, solch einem Projekt diehnt als Ablageort für Daten über 
\begin{itemize}
\item absolvierte Analysevorgänge
\item relevante Projekteinstellungen
\item vom Benutzer bereitgestellte Resourcen (z.B. ein Video)
\end{itemize}
\nItem{MK} Die Anwendung kann Videos im \gls{glos:YUV} Format von der Festplatte lesen, manipulieren und wieder auf die Festplatte schreiben.

\section{Wunschkriterien}
\setcounter{enumi}{0}
\nItem{WK} Die Daten die der der \gls{glos:VE} vom \gls{ITEC}  bereitstellt sollen visuell aufbereitet werden
\nItem{WK} Der Nutzer sollte ohne langer Einarbeitungsphase das Programm bedienen können.

\nItem{WK} Das nachträgliche portieren der Anwendung in eine andere Sprache sollte möglichst einfach sein.

\nItem{WK} Ein angelegtes Projekt ist auch unter einer anderen \projektTitel Installation lauffähig, vorausgesetzt die nötigen Resourcen (z.B. Video Dateien die dem Projekt zuvor zur Verfügung gestellt wurden) stehen bereit.

\section{Abgrenzungskriterien}
\setcounter{enumi}{0}
\nItem{AK} Audiounterstützung ist nicht geplant
\nItem{AK} \projektTitel verfolgt nicht das Ziel ein Videoverarbeitungswerkzeug auf seine Performanz zu untersuchen.
\nItem{AK} Da die Videoverarbeitungswerkzeuge nicht direkt in die zu entwickelnde Anwendung eingebunden werden sind keine Echtzeitanalysen vorgesehen.

