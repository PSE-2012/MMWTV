\chapter{Zielbestimmung}

\section{Mußkriterien}

\newItemMK Nach einem Analysedurchlauf soll dem Nutzer die Möglichkeit angeboten werden, die entstandenen Daten zu exportieren. %1
\newItemMK Ein gegebenes Video im YUV Format kann mit verschiedenen Testsignalen verfälscht werden %2
\newItemMK Die Anwendung sollte ohne großen Aufwand um neue Filter oder Bewertungsmetriken erweitert werden können. %3
\newItemMK Die Ergebnisse eines Analysedurchlaufs sollen Graphisch aufbereitet werden. %4
\newItemMK Viele Standarttestsignale/Filter/Bewertungsmetriken sollen der Anwendung bereits zum Auslieferungszeitpunkt zur Verfügung stehen %5
\newItemMK Die Anwendung soll auf einem im Handel erwerbbaren Rechner lauffähig sein. %6

\newItemMK Benutzer haben die Möglichkeiten Projekte anzulegen %7

\newItemMK Die Anwendung kann Videos im .YUV Format von der Festplatte laden, manipulieren und wieder auf die Festplatte schreiben. %8


\setcounter{enumi}{0}

\section{Wunschkriterien}

\newItemWK Die Daten die der InstitutsEncoder bereitstellt sollen visuell aufbereitet werden
\newItemWK Der Endnutzer sollte ohne aufwendiger Einarbeitung das Programm bedienen können.
\newItemWK Analyseergebnisse können zu einem bestimmten Videoverarbeitungswerkzeug zugeordnet und chronologisch abgespeichert werden.
\newItemWK Das Programm ist in der Lage .YUV Video mit verschiedenen Testsignalen zu verschmelzen.
\newItemWK Der Nutzer hat die Möglichkeit die Ergebnisse eines Analysedurchlaufs zu exportieren.
\newItemWK Neue Filter/Metriken können auch nach Auslieferung des Programms zur Anwendung hinzugefügt werden. Dazu
  sind keine Änderungen am der Anwendung selbst notwendig
\newItemWK Das nachträgliche portieren der Anwendung in eine andere Sprache sollte möglichst einfach sein.

\newItemWK Der Nutzer soll die möglichkeit haben seine erstellten Projekte auf jedem anderen Computer nutzen zu können.

\setcounter{enumi}{0}

\section{Abgrenzungskriterien}

\newItemAK Audiounterstützung ist nicht geplant
\newItemAK keine Analysierung der Leistungsdaten
\newItemAK Da die Videoverarbeitungswerkzeuge nicht direkt in die zu entwickelnde Anwendung eingebunden werden sind keine Echtzeitanalysen vorgesehen.


\setcounter{enumi}{0}
