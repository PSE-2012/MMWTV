%so wird das glossary indexiert
%pdflatex datei
%makeindex -s datei.ist -t datei.alg -o datei.acr datei.acn
%makeindex -s datei.ist -t datei.glg -o datei.gls datei.glo
%makeindex -s datei.ist -t datei.slg -o datei.syi datei.syg
%pdflatex datei

\newglossary[slg]{symbolslist}{syi}{syg}{Symbolverzeichnis}
\renewcommand*{\glspostdescription}{}
\makeglossaries

%Syntax um Symbole zu definieren
%\newglossaryentry{symb:Pi}{
%name=$\pi$,
%description={Die Kreiszahl.},
%sort=symbolpi, type=symbolslist
%}
%mit \gls{symb:Pi} wird dann das Symbol gesetzt

%Syntax um Abkürzungen zu definieren
%\newacronym{MS}{MS}{Microsoft}
%%Eine Abkürzung mit Glossareintrag
%\newacronym{AD}{AD}{Active Directory\protect\glsadd{glos:AD}}
%mit z.B. \gls{MS} wird die Abkürzung im Text eingebunden
\newacronym{qos}{QOS}{Quality of Service\protect\glsadd{glos:qos}}
\newacronym{mos}{MOS}{Mean opinion Score\protect\glsadd{glos:mos}}
\newacronym{mse}{MSE}{Mean squared error\protect\glsadd{glos:mse}}
\newacronym{snr}{SNR}{Signal-to-noise ratio\protect\glsadd{glos:snr}}
\newacronym{psnr}{PSNR}{Peak signal-to-noise ratio\protect\glsadd{glos:psnr}}
\newacronym{FR}{FR}{Full Reference\protect\glsadd{glos:FR}}
\newacronym{csv}{.CSV}{comma separated value}
\newacronym{KIT}{KIT}{Karlsruher Institut für Technologie}
\newacronym{ITEC}{ITEC}{Institut für technische Informatik}
\newacronym{CES}{CES}{Chair for Embedded Systems}
\newacronym{OQAT}{OQAT}{Objective Quality Assessment Toolkit}
\newacronym{fps}{FPS}{Frames per second}
\newacronym{MSE}{MSE}{Mean Square Error}
\newacronym{PSNR}{PSNR}{Peak signal-to-noise ratio}
\newacronym{VBW} {VBW} {Videobearbeitungswerkzeug}
%Syntax um Glossareinträge zu definieren
%\newglossaryentry{glos:AD}{
%name=Active Directory,
%description={Active Directory ist in einem Windows 2000/" "Windows
%Server 2003-Netzwerk...}
%}
%mit \gls{glos:AD} kann man es im Text einbinden
\newglossaryentry{glos:psnr}{
name=Peak signal-to-noise ratio,
description={!!Ich wurde noch nicht beschrieben!!}}
\newglossaryentry{glos:snr}{
name=Signal-to-noise ratio,
description={!!Ich wurde noch nicht beschrieben!!}}
\newglossaryentry{glos:mse}{
name=Mean squared error,
description={!!Ich wurde noch nicht beschrieben!!}}
\newglossaryentry{glos:qos}{
name=Quality of Service,
description={!!Ich wurde noch nicht beschrieben!!}}
\newglossaryentry{glos:mos}{
name=Mean opinion score,
description={!!Ich wurde noch nicht beschrieben!!}}
\newglossaryentry{glos:FR}{
name=Full Reference quality assessment,
description={!!Ich wurde noch nicht beschrieben!!}}
\newglossaryentry{glos:VBW}{
name=Videobearbeitungswerkzeug,
description={Ein Werkzeug, dessen Qualität getestet werden soll; z.B. ein Video Encoder, aber auch Bildbearbeitungswerkzeuge sind möglich}}


\newglossaryentry{glos:Filter}{
name=Filter,
description={Ein Werkzeug, mit dem \projektTitel ein Video manipuliert, um anschließend bestimmte Eigenschaften eines \gls{VBW} gezielt auf diesem Video zu testen.}}


% TODO: Wir sollten uns einigen, wofür wir eigentlich diese Begriffe genau verwenden.
\newglossaryentry{glos: Rd}{
name=Referenzdatei / Referenzvideo,
description={!!Ich wurde noch nicht beschrieben!!}}
\newglossaryentry{glos: Tv}{
name=Testdatei / Testvideo,
description={!!Ich wurde noch nicht beschrieben!!}}


\newglossaryentry{glos:VE}{
name=Video Encoder,
description={!!Ich wurde noch nicht beschrieben!!}}

\newglossaryentry{glos:YUV}{
name=.YUV,
description={!!Ich wurde noch nicht beschrieben!!}}
