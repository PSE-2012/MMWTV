%so wird das glossary indexiert
%pdflatex datei
%makeindex -s datei.ist -t datei.alg -o datei.acr datei.acn
%makeindex -s datei.ist -t datei.glg -o datei.gls datei.glo
%makeindex -s datei.ist -t datei.slg -o datei.syi datei.syg
%pdflatex datei

\newglossary[slg]{symbolslist}{syi}{syg}{Symbolverzeichnis}
\renewcommand*{\glspostdescription}{}
\makeglossaries

%Syntax um Symbole zu definieren
%\newglossaryentry{symb:Pi}{
%name=$\pi$,
%description={Die Kreiszahl.},
%sort=symbolpi, type=symbolslist
%}
%mit \gls{symb:Pi} wird dann das Symbol gesetzt

%Syntax um Abkürzungen zu definieren
%\newacronym{MS}{MS}{Microsoft}
%%Eine Abkürzung mit Glossareintrag
%\newacronym{AD}{AD}{Active Directory\protect\glsadd{glos:AD}}
%mit z.B. \gls{MS} wird die Abkürzung im Text eingebunden

\newacronym{KIT}{KIT}{Karlsruher Institut für Technologie}
\newacronym{ITEC}{ITEC}{Institut für technische Informatik}
\newacronym{CES}{CES}{Chair for Embedded Systems}

\newacronym{MSD}{MSD}{Mean Square Difference}
\newacronym{PSNR}{PSNR}{Peak signal-to-noise ratio}

%Syntax um Glossareinträge zu definieren
%\newglossaryentry{glos:AD}{
%name=Active Directory,
%description={Active Directory ist in einem Windows 2000/" "Windows
%Server 2003-Netzwerk...}
%}
%mit \gls{glos:AD} kann man es im Text einbinden

\newglossaryentry{glos:VBW}{
name=Videobearbeitungswerkzeug,
description={Ein Werkzeug, dessen Qualität getestet werden soll; z.B. ein Encoder, aber auch Bildbearbeitungswerkzeuge sind möglich}}

% TODO: Wir sollten uns einigen, wofür wir eigentlich diese Begriffe genau verwenden.
\newglossaryentry{glos: }{
name=Referenzdatei / Referenzvideo,
description={!!Ich wurde noch nicht beschrieben!!}}
\newglossaryentry{glos: }{
name=Testdatei / Testvideo,
description={!!Ich wurde noch nicht beschrieben!!}}


\newglossaryentry{glos:VE}{
name=Video Encoder,
description={!!Ich wurde noch nicht beschrieben!!}}

\newglossaryentry{glos:YUV}{
name=.YUV,
description={!!Ich wurde noch nicht beschrieben!!}}
