\chapter {Produktfunktionen}

\subsection{Grundfunktionen}
\setcounter{counterKriterien}{0}
\nItem{PF} Willkommensbildschirm
\newline
Beim Start zeigt die Anwendung einen Willkommensbildschirm an, auf dem allgemeine Informationen und zuletzt benutzte Projekte zusammen mit einer \emph{Projekt-Erstellung-Sicht} angezeigt werden.

\nItem{PF} Projektverwaltung
\newline
Die Anwendung bietet dem Benutzer die Möglichkeit, durch eine \emph{Projekt-Erstellung-Sicht} Projekte anlegen, speichern, löschen und verwalten zu können.

\nItem{PF} Laden von Referenzdateien
\newline
Der Benutzer kann durch ein Dialogfenster eine Referenzdatei im YUV-Format in ein Projekt laden, indem er den Pfad zu der Datei angibt. Es können mehrere Referenzdateien geladen werden.

\nItem{PF} Projektexplorer
\newline
Die Anwendung listet alle Referenzvideos und Testvideos des Projekts in einem Projektexplorer auf. Der Benutzer kann ein oder mehrere Videos auswählen um auf diesen Filter oder Analysen auszuführen.



\nItem{PF} Filter
\newline
Ein Video kann durch Filter verfälscht werden um ein Testvideo zu generieren, mit dem bestimmte Eigenschaften des \gls{glos:VBW} gezielt analysiert werden sollen.

\nItem{PF} Standard-Filter
\newline
Die Anwendung stellt dem Benutzer einige Filter standardmäßig bereit:
\begin{itemize}
\item Weichzeichner
\item ...
\end{itemize}

\nItem{PF} Filter-Plugins
\newline
Die Anwendung ist um beliebige Testsignale und Filter in der Form von Plugins erweiterbar.


% Sind diese Ansteuerungen von den Encodern/Tools die getestet werden nicht letztlich so etwas wie "Filter"?
% Auch sie verfälschen ja ein Referenzvideo - und am Ende ist es unserem Programm egal, wie die Videos entstanden sind, die es dann vergleicht.
\nItem{PF} Manuelle Ansteuerung eines \gls{glos:VBW}
\newline
Ein zu testendes \gls{glos:VBW} kann manuell angesteuert werden. Dabei wird der Benutzer zu Beginn der Analysephase aufgefordert, anzugeben wo sich die zu den Referenz-/Testdateien gehörigen Dateien befinden, die durch das \gls{glos:VBW} bearbeitet wurden.
\nItem{PF} Automatische Ansteuerung eines \gls{glos:VBW}
\newline
Ein zu testendes \gls{glos:VBW} kann automatisch angesteuert werden. Dazu macht der Benutzer in der \emph{Projekt-Erstellung-Sicht} nötige Angaben zu Programmpfad und Parametern. \projektTitel steuert dieses \gls{glos:VBW} dann zu Beginn der Analysephase automatisch an und generiert die entsprechenden Vergleichsdateien für die Analyse.



\nItem{PF} Testsignalansicht
\newline
Alle Filter und Testsignale, mit denen ein Referenzdatei verfälscht werden kann, werden aufgelistet. Der Benutzer kann beliebige Filter auswählen und nacheinander auf eine bestehende Datei anwenden. Über einen Button kann die \emph{Testsignalauswahlphase} abgeschlossen werden.

\nItem{PF} Generierung von Testdateien
\newline
Nach Abschluss der \emph{Testsignalauswahlphase} werden die ausgewählten Filter auf die Referenzdatei angewendet und eine verfälschte Testdatei im YUV Format generiert, die im Projektordner gespeichert wird. Somit ist die \emph{Generierungsphase} abgeschlossen.





\nItem{PF} Analysewerkzeugsicht
\newline
Nach Abschluss der \emph{Generierungsphase} wird eine \emph{Analysewerkzeugsicht} automatisch angezeigt. Über einen Button kann man auch manuell für zwei Referenzvideos direkt eine \emph{Analysewerkzeugsicht} öffnen, ohne die \emph{Testauswahlphase} und die \emph{Generierungsphase} durchzugehen. Alle verfügbaren Analysewerkzeuge werden in dieser Sicht aufgelistet. Der Benutzer kann beliebige Werkzeuge markieren und schliesslich über einen Button einen Analysedurchlauf starten.

\nItem{PF} Analyse-Metriken
\newline
\projektTitel stellt dem Benutzer einige Analyse-Metriken standardmäßig bereit:
\begin{itemize}
\item \gls{MSE}
\item \gls{PSNR}
\end{itemize}


\nItem{PF} Analysedurchlauf
\newline
Falls im Projekt angegeben wurde, dass das zu testende Videobearbeitungswerkzeug manuell angesteuert wird, 
bekommt der Nutzer ein Dialogfenster angezeigt, in dem er den Pfad zu dem vom Werkzeug manuell verarbeiteten Video angeben muss, das dann mit dem Referenzvideo verglichen werden soll. Falls manuelle Steuerung nicht angegeben wurde, sollen die zwei Referenzvideos bzw. das Referenzvideo und die von unserer Anwendung verfälschte Version davon verglichen werden. Dann startet der Analysedurchlauf. Die zwei angegebenen Videodateien werden mit einander mittels der ausgewählten Analysemetriken verglichen und die Ergebnisse analysiert, bewertet und optional in eine Logdatei gespeichert. Dem Nutzer werden während des Analysedurchlaufs ein Fortschrittsbalken und Statusmeldungen angezeigt. 

\nItem{PF} Darstellung der Analyseergebnisse als Diagramme
\newline
Die Analysedaten werden graphisch durch Diagramme dargestellt.

\nItem{PF} Darstellung der Analyseergebnisse als Overlay
\newline
Analysedaten (z.B. berechnete Unterschiede) werden als Overlay über dem Video eingeblendet.


\nItem{PF} GUI
\newline
Die Anwendung unterstützt eine interaktive graphische Benutzeroberfläche mit verschiedenen Sichten, die ein- und ausgeblendet werden können. Dabei können Ein- und Ausgabevideos abgespielt, sowie einzelne Frames ausgewählt und visuallisiert werden.





\subsection{Optionale Funktionalität}

\nItem{PF} Vorschau für Filter
\newline
Die Auswirkungen eines Filters oder Testsignals auf das gewählte Video wird dem Benutzer schon in der \emph{Testsignalauswahlphase} als Vorschau an einem Frame veranschaulicht.


\nItem{PF} Eingebaute Kommandozeile
\newline
Der Benutzer hat die Möglichkeit, durch eine Kommandozeile Argumente an die Anwendung übergeben zu können.

\nItem{PF} Logdatei
\newline
Es gibt die Option, eine Logdatei für einen Analysevorgang zu erstellen. In der Logdatei sollen im Rahmen des jeweiligen Analysevorgangs ausgeführte Operationen sowie erhaltene Ergebnisse in Textformat chronologisch aufgelistet werden. Die Auswahl, welche Arten von Operationen und Ergebnissen in der Logdatei gespeichert werden sollen, steht dem Benutzer zur verfügung.
