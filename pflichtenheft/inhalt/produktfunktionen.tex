\chapter {Produktfunktionen}

\subsection{Grundfunktionen}
\setcounter{counterKriterien}{0}
\nItem{PF} Willkommensbildschirm
\newline
Beim Start zeigt die Anwendung einen Willkommensbildschirm an, auf dem allgemeine Informationen und zuletzt benutzte Projekte zusammen mit einer \emph{Projekt-Erstellung-Sicht} angezeigt werden.

\nItem{PF} Projektverwaltung
\newline
Die Anwendung bietet dem Benutzer die Möglichkeit, durch eine \emph{Projekt-Erstellung-Sicht} Projekte anlegen, speichern, löschen und verwalten zu können.

\nItem{PF} Laden von Referenzdateien
\newline
Der Benutzer kann durch ein Dialogfenster eine Referenzdatei im YUV-Format in ein Projekt laden, indem er den Pfad zu der Datei angibt. Es können mehrere Referenzdateien geladen werden.

\nItem{PF} Projektexplorer
\newline
Die Anwendung listet alle Referenzvideos und Testvideos des Projekts in einem Projektexplorer auf. Der Benutzer kann ein oder mehrere Videos auswählen um auf diesen Filter oder Analysen auszuführen.



\nItem{PF} Filter
\newline
Ein Video kann durch Filter verändert werden um ein Testvideo zu generieren, mit dem bestimmte Eigenschaften des \gls{glos:VBW} gezielt analysiert werden sollen.

\nItem{PF} Standard-Filter
\newline
Die Anwendung stellt dem Benutzer einige Filter standardmäßig bereit:
\begin{itemize}
\item Weichzeichner
\item ...
\end{itemize}

\nItem{PF} Filter-Plugins
\newline
Die Anwendung ist um beliebige Testsignale und Filter in der Form von Plugins erweiterbar. Videobearbeitungswerkzeuge (wie z. B. Encoder) kann man auch als externe Filter betrachten und sie als Plugins laden, solange sie die Form von ausführbaren Programmen haben.

\nItem{PF} Testsignalansicht
\newline
Alle Filter und Testsignale, mit denen eine Referenzdatei manipuliert werden kann, werden aufgelistet. Der Benutzer kann beliebige Filter auswählen und nacheinander über einen Button auf eine bestehende Datei anwenden.

\nItem{PF} Generierung von Testdateien
\newline
Wendet man ausgewählte Filter auf eine Datei an, dann wird eine durch die jeweiligen Filter manipulierte Datei im YUV Format generiert, die im Projektordner gespeichert und im Projektexplorer bei den Testvideos aufgelistet wird.

\nItem{PF} Analysewerkzeugsicht
\newline
Alle verfügbaren Analysewerkzeuge werden in dieser Sicht aufgelistet. Der Benutzer kann beliebige Werkzeuge markieren und schliesslich über einen Button einen Analysedurchlauf für die aus dem Projektexplorer ausgewählten Videodateien starten.

\nItem{PF} Analyse-Metriken
\newline
\projektTitel stellt dem Benutzer einige Analyse-Metriken standardmäßig bereit:
\begin{itemize}
\item \gls{MSE}
\item \gls{PSNR}
\end{itemize}


\nItem{PF} Analysedurchlauf
\newline
Wenn ein Analysedurchlauf gestartet wird, werden die ausgewählten Videos mittels der ausgewählten Analysemetriken verglichen und die Ergebnisse analysiert, bewertet und optional in eine Logdatei gespeichert. Dem Nutzer werden während des Analysedurchlaufs ein Fortschrittsbalken und Statusmeldungen angezeigt. 

\nItem{PF} Darstellung der Analyseergebnisse als Diagramme
\newline
Die Analysedaten werden graphisch durch Diagramme dargestellt.

\nItem{PF} Darstellung der Analyseergebnisse als Overlay
\newline
Analysedaten (z.B. berechnete Unterschiede) werden als Overlay über dem Video eingeblendet.


\nItem{PF} GUI
\newline
Die Anwendung unterstützt eine interaktive graphische Benutzeroberfläche mit verschiedenen Sichten, die ein- und ausgeblendet werden können. Dabei können Ein- und Ausgabevideos abgespielt, sowie einzelne Frames ausgewählt und visuallisiert werden.





\subsection{Optionale Funktionalität}

\nItem{PF} Vorschau für Filter
\newline
Die Auswirkungen eines Filters oder Testsignals auf ein gewähltes Video werden dem Benutzer als Vorschau an einem Frame veranschaulicht.


\nItem{PF} Eingebaute Kommandozeile
\newline
Der Benutzer hat die Möglichkeit, durch eine Kommandozeile Argumente an die Anwendung übergeben zu können.

\nItem{PF} Logdatei
\newline
Es gibt die Option, eine Logdatei für einen Analysevorgang zu erstellen. In der Logdatei sollen im Rahmen des jeweiligen Analysevorgangs ausgeführte Operationen sowie erhaltene Ergebnisse in Textformat chronologisch aufgelistet werden. Die Auswahl, welche Arten von Operationen und Ergebnissen in der Logdatei gespeichert werden sollen, steht dem Benutzer zur verfügung.
