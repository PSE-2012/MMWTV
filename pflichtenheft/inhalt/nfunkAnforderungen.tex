\chapter{Nichtfunktionale Anforderungen}
\section{Zuverlässigkeit}
\setcounter{counterKriterien}{0}
\nItem{NF} Fehlerhafte Eingaben sollen durch die GUI weitestgehend vermieden werden.
\nItem{NF} Fehlerhafte Eingaben werden erkannt und mit entsprechender Meldung geahndet.
\nItem{NF} Fehlerhaft eingelesene Videos sollen nicht zum abstürzten führen.
\nItem{NF} Projekte müssen verschiedene Namen haben, sonst wird der Benutzer aufgefordert einen Anderen einzugeben.
%\section{Aussehen und Handhabung} %siehe GUI
\section{Benutzbarkeit}
\begin{itemize}
% Und was kosten Kekse ?
% Eine benutzerfreundliche auslagerung ist eine ungefähr so sinnvolle Formullierung, im hinblick
% auf den Informationsgehalt, wie meine Frage nach dem Preis eines Keks ;-) .
\item Das Programm wird Benutzerfreundlich ausgelegt.
%Der Nutzer sollte ohne langer Einarbeitungsphase das Programm bedienen können.
\item \projektTitel wird unerfahrenen Benutzern eine Hilfe anbieten.
\end{itemize}
\section{Leistung und Effizienz}
% Ich will das löschen.
\begin{itemize}
\item Es sind keine Echtzeitanalysen vorgesehen, trotzdem wird die Laufzeit von Videoverarbeitungswerkzeuge durch \projektTitel nicht exponentiell gesteigert.
\end{itemize}
\section{Wartbarkeit und Änderbarkeit}
\begin{itemize}
%Das steht ja immernoch da.. hab ich nicht mal deswegen ein Issue aufgemacht und es hieß man hat es
%bereinigt ?
\item Das nachträgliche portieren der Anwendung in eine andere Sprache sollte möglichst einfach sein.
\item Durch Erweiterungsmodule sind Filter und Bewertungsmetriken leicht zu ergänzen.
\end{itemize}