\chapter{Einleitung}

Die Anwendung "Objective Quality Assessment Toolkit", welche im Auftrag des CES hergestellt wird, wird wie im Pflichtenheft angekündigt nach dem "Modell View Viewmodell" Entwurfsmuster angefertigt.
Hierbei ist das Ziel eine durchgehende Trennung der Aufgaben zu erreichen.

Das Subsystem Modell enthält alle statischen Daten, wie Videos und Projekte. Außerdem bietet es zusätzlich die Möglichkeit an, diese auf die Festplatte zu schreiben. Hier liegen Auserdem die Plugins, was Filter, Metriken sowie die Komponenten zum Umgang mit Videos beinhaltet.

Das ViewModell ist die Schnittstelle zwischen Modell und View. Es ist  dafür zuständig erhaltene Eingaben des Benutzer zu verarbeiten und diese gegebenenfalls an das Modell weiterzureichen. Jeder Viewkomponente ist ein Viewmodelll zugeordnet.

Die View zeigt den Aktuellen Zustand des Modells an. Da sich die Viewkomponenten nicht Sinvoll im Entwurf darstellen lassen, wurde auf sie im Klassendiagramm verzichtet.