\chapter{Einleitung}

Die Anwendung "Objective Quality Assessment Toolkit", welche im Auftrag des CES hergestellt wird, wird wie im Pflichtenheft angekündigt nach dem "Modell View ViewModell" Entwurfsmuster angefertigt.
Hierbei ist das Ziel eine durchgehende Trennung der Aufgaben zu erreichen.

Das Subsystem Modell enthält alle statischen Daten, wie Videos und Projekte. Darüber hinaus wird die  Möglichkeit diese auf die Festplatte zu schreiben realisiert. Auserdem werden die Plugins, was Filter, Metriken sowie die Komponenten zum Umgang mit Videos beinhaltet im Modell abgelegt.

Das ViewModell ist die Schnittstelle zwischen Modell und View. Es ist  dafür zuständig erhaltene Eingaben des Benutzer zu verarbeiten und diese gegebenenfalls an das Modell weiterzureichen. Jeder View-Komponente ist ein ViewModell zugeordnet.

Die View zeigt den Aktuellen Zustand des Modells an. Da sich die View-komponenten nicht Sinnvoll im Entwurf darstellen lassen, weil sie im XAML Code stehen, wurde auf sie im Klassendiagramm verzichtet.