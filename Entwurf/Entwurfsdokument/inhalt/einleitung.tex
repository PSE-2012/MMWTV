\chapter{Einleitung}

Die Anwendung \emph{Objective Quality Assessment Toolkit} , welche im Auftrag des CES hergestellt wird, wird wie im Pflichtenheft spezifiziert nach dem \emph{Model View ViewModel} Entwurfsmuster angefertigt.
Hierbei ist das Ziel eine möglichst lose gekoppeltes System der Aufgaben zu erreichen.

Das ViewModel fungiert unter anderem als koordinationsmodul
zwischen View und Model, so werden z.b. Databindings welche
sich nicht deklarativ (zur Entwicklungszeit im XAML-code)
konstruieren lassen von einem ViewModel einer jeweiligen View
an ein bestimmtes ModelElement gebunden.

Das Model übernimmt neben der Datenhaltung und Organisation
auch das Sichern bzw. Wiederfinden der Daten auf der Festplatte.
Wobei Daten \projektTitel-Anwendungseinstellungen (Sprache, bestimmte
Standartpfade), von \projektTitel oder
aber von einem Plugin von \projektTitel erstellte Videoinformationen.

Da die View für \projektTitel mit Hilfe von WPF(Win...) entwickelt 
wird ist die Verwendung von deklarativen Mitteln (XAML) den
imperativen (C sharp) vorzuziehen. Durch solches vorgehen wird
codedopplung bei der GUI-Entwicklung vermieden und eine
gewisse Robustheit erreicht. Ein Nachteil der weites gehenden deklarativen
Programmierung der View ist der Verlust einer Sinnvollen
UML-Diagramm-Darstellungen für diese ohne mit der Implementierung 
der View anzufangen, daher wurde auf das Erstellen einzelner View-Klassen
verzichtet. Um dennoch eine Vorstellung für die zu entwickelnde GUI 
zu bekommen, kann das Pflichtenheft zur Rate gezogen werden.
