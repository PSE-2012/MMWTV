\chapter{System und Integrationstests}
\section{Qualitätsbestimmung}
\begin{itemize}
\item /Q-10/ Die GUI soll falsche Benutzereingaben weitestgehend vermeiden. \newline
Der Punkt wurde durch einsatz von komponenten, welche solche eingaben vermeiden(z.B.Slider)wo möglich erfüllt.
\item /Q-20/ Fehlerhafte Eingaben für Pfad, Frames per second, Filter und Analysemetriken werden vom Programm nicht angenommen und der Benutzer wird gefordert, sie zu korrigieren. \newline
Wurde durch Messageboxen realisiert.
\item /Q-30/ Ein Filter- oder Analysevorgang findet nur für gültige YUV-Dateien statt, d.h. wenn eine YUV-Datei nicht gelesen werden kann, stürzt das Programm nicht ab und gibt eine entsprechende Fehlermeldung aus. \newline
Die Dateien werden vor dem hinzufügen auf Gültigkeit getestet.
\item  /Q-40/Oqat wird Benutzern eine Hilfe, in Form von Tooltipps und einer kurzen Einleitung die wichtigsten Programmfunktionen anbieten. \newline
Es wurde eine Dokumentation über die einzelnden Ansichten angefertigt.
\item Die Anwendung wird ausführlich getestet. \newline
ähm ... ja
\end{itemize}]
\section{Testfälle}
\begin{itemize}
\item /T-10/ Projekt erstellen, speichern und laden
funktioniert
\item /T-20/ Projekt bearbeiten, speichern und laden
Videos lassen sich hinzufügen/löschen und verändern.
\item /T-30/ Projekt auf anderem Rechner öffnen
Wenn die Pfadressourcen stimmen. Ansonsten gehen die Videos Verloren.
\item /T-40/ Filter auswählen ohne Video ausgewählt zu haben - Fehlermeldung
Funktioniert
\item /T-50/ Filter auswählen, Vorschau betrachten
Nicht Implementiert
\item /T-60/ Filter anwenden, generierte Video-Datei überprüfen

\item /T-70/ Filter-Einstellungen verändern
Funktioniert über Properties View bzw über Memento-Funktionalität.
\item /T-80/ Mehrere Filter auf ein Video anwenden, Reihenfolge verändern
Funktioniert per Drag-and-Drop
\item /T-90/ Analysemetrik auswählen ohne zwei Videos ausgewählt zu haben - Fehlermeldung
Funktioniert
\item /T-100/ Analyse starten ohne Metrik auszuwählen - Fehlermeldung
Funktioniert
\item /T-110/ Analyse durchführen, Ergebnisse anzeigen
Funktioniert(stand alter macro)
\item /T-120/ Analyseergebnisse speichern und laden
Funktioniert(stand alter macro)
\item /T-130/ Analyseergebnisse exportieren (CSV)
Funktioniert(stand alter macro)
\item /T-140/ GUI Funktionalität
Funktioniert
\end{itemize}