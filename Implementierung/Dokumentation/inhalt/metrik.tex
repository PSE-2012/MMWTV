\chapter{Metrikenansicht}
\includegraphics[scale=0.55]{bilder/Metriken.png}\\[5ex]
Die Metrik-Ansicht unterscheidet sich von der Filter-Ansicht nur dadurch, dass nun zwei Player angezeigt werden und in die Macroliste jetzt ">Metriken"< in eine Schlange eingereiht werden können.
Wie bei den Filtern können eigene Metriken hinzugefügt werden, indem eine DLL-Datei mit dem Plugin in den Ordner ">Plugins"< im gleichen Verzeichnis wie die OQAT-Exe gelegt wird.


\begin{itemize}
\item MSE \newline
Je ähnlicher die Farben der Bilder sich sind, desto niedriger ist der wert. 0 bedeutet, dass die Bilder identisch sind.
\item PSNR \newline
Sind die einzelnen Farbkanäle ähnlich, resultiert ein niedriger Wert. Da es sich um eine logarithmische Funktion handelt wird 0 nicht erreicht. Stattdessen wird bei gleichen Bildern der Wert -1 ausgegeben.
\item Eigene Metriken \newline
Es können eigene Metriken als Plugin hinzugefügt werden, indem eine DLL-Datei mit dem Plugin in den Ordner ">Plugins"< im gleichen Verzeichnis wie die OQAT-Exe abgelegt wird. Die zu implementierenden Interfaces IPlugin und IMetricOqat sind der OQAT-API zu entnehmen.
\end{itemize}
