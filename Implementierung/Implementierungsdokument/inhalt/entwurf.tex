\chapter{Änderungen am Entwurf}
\begin{itemize}
\item Viele Klassen implementieren nun ISerilizable, damit Mementos abgespeichert werden können.
\item Smartnode \newline
Es wurden die Attribute name, id und fatherId hinzugefügt. Außerdem wurde der Konstruktor geändert, da es nun einfacher ist Kinder hinzuzufügen. Er nimmt nun ein Video, id und optional einen Smarttree als Argumente.
\item Caretaker \newline
Die Struktur wurde Kompakter, für größere Dynamik und weniger organisatorischen Aufwand.
\item Project \newline
Hinzugefügt wurde: Attribut unusedId um ungenuzte Werte zu finden, saveProject Methode und Methoden zum Hinzufügen und Löschen von Smartnodes. Gelöscht wurde die Memento funktionalität, sowie die Konstruktoren, welche eine Videoliste bzw. nur einen Pfad als Argument benötigten.
\item Video \newline
Die Mementofunktionalität wurde entfernt. Hinzugefügt wurde eine Liste der Plugins, die das Video bearbeitet haben, sowie die getExtraHandler Methode.
\item VideoSource \newline
Wurde entfernt, da nicht benötigt.
\item OqatSerializationBinder \newline
Wurde Hinzugefügt.
\item class MementoListMemento \newline
Wurde hinzugefügt um alle Mementos eines Objektes in einem einzigen File zu speichern.
\item PluginManager: \newline
Der Typ der Plugintable wurde auf Enumerable geändert, da es praktischer ist. getPluginmanager, wurde in den getter des Pluginmanagers verschoben, da es besser aussieht und weniger Code ist.
Die Methode loadPluginTable wurde gelöscht, da ihre Funktion automatisch abläuft. Außerdem wurde die Hanldertable gelöscht.
\item Macro \newline
Wurde neu designed.
\item Oqat \newline
Event onNewProjectCreated wurde hinzugefügt und Initialisierungsmethoden wurden gelöscht, da nicht verwendet.
\item Pluginlist \newline
Es wurde CurrentViewType und das Pluginmanger Attribut entfernt. Der Konstruktor nimmt nun den PluginTyp als Argument.
\item ProjectExplorer \newline
Pluginmanger Attribut wurde gelöscht und importVideos Methode hinzugefügt, um durch den Video import Dialog zu führen.
\item Welcome \newline
Konstruktor nimmt kein Panel mehr an.
\item VidImportOptionsDialog \newline
Nimmt nun eine Liste von Videos an um mehrere Videos zu importieren.
\item WindowsErrorConsole \newline
implementiert zum Debugging.
\item WindowOqatInfo \newline
Implementiert um allgemeine Informationen über \projektTitel anzuzeigen.
\item Sandbox \newline
Klasse lädt temporär ein Plugin für Debugging-Zwecke.
\item EntryEventArgs \newline
Implementiert um String in eventArgs zu verpacken.
\item PluginType \newline
Einträge umbenannt zum jeweiligen Interface Namen, da es praktischer ist um an das jeweilige Gegenstück zu kommen.
\item EventType \newline
Wurde um neue Eventypen erweitert.
\item IVideoInfo \newline
Implementiert nun IClonable und hat drei zusätzlichen Attribute width, height und framecount.
\item MetaData \newline
Plugins müssen dieses Interface nun nützen um richtig geladen zu werden.
\item IVideo \newline
Wurde hinzugefügt.
\item IPlugin \newline namePlugin und type Attribute wurden gelöscht und werden nun in IPluginMetadata bereitgestellt. propertiesView Attribut wurde hinzugefügt sowie die createExtraInstacne Methode, falls man ein weiteres Objekt des Typs braucht. Ungenutzte Methoden wurden entfernt.
\item IPresentation \newline
Wurde wegen Player und Handler neu designed.
\item IMacro \newline
Implementiert nun IPlugin, da sonst der Pluginmanager das Interface nicht erkennen kann.
\item Player \newline
Neu designed, auf Grund Performance Problemen.
\item YuvVideoHandler \newline
Neu designed, auf Grund Performance Problemen.
\item Plugins \newline
Minor changes with setVideo. SSIM wurde nicht realisiert, da der Wertebereich falsch war.

\end{itemize}


